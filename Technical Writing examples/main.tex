\documentclass{book}
\usepackage{graphicx} % Required for inserting images

\title{Huw's Big Book of Linux}
\author{Huw Coverdale Jones}
\date{April 2023}

\begin{document}

\maketitle
\tableofcontents


\chapter{Introduction - What is
Linux and Why Use it?}

\section{What is it?}
Once upon a time, there was a period where Windows may have faced some competition as the main Operating system for computers. Apple was broadly affordable, at least at primary schools in North Lincolnshire, and UNIX was operating on several systems. The NeXT Cube was running an OS made by Stephen Jobsworth (colloquially known by his stage name, Steve Jobs), and that was used by Tim Berners-Lee to invent the world wide web protocol, leading to the social phenomenon known by many as the internet. This means Twitter is his fault by the way.

Among this brave new multiple OS world, there lived in Fenno-Swede geek named Linus Torvalds. In his immortal words he said "I'm doing a (free) operating system (just a hobby, won't be big and professional like gnu) for 386(486) AT clones." After this, he took the basics of UNIX, made his own little system, and years later caused self indulgent geeks to be even less welcome at parties. 

Since then Linux has spiraled into many directions, all of which are definitely positively correlated with having lots of friends and being sexually active. Linux Mint and Ubuntu serve as free versions of MacOS, while Kali Linux is for people who hack for a living, or have just seen Mr Robot. Tails OS, Porteus, and Slack serve as computers that you can carry in your pocket on a USB stick. This, of course relies on finding a computer you can put said USB into, but even geeky past-times require marketing jargon these days.

For more technical explanations of what it is, use google. It's literally a free computer system that anyone can download, there's tons of info. Now, why use it?

\section{Why use it?}

There are many reasons to use Linux, just ask your local Games Workshop employee. The main ones are simple; it is often more lightweight than its commercial rivals, can be adapted to fit many functions, and it's pretty much always free. 

So, the first point; "lightweight" means "doesn't have too much bloated stuff that slows  down your hardware," in this case. This is far from universally true to Linux. For example, Kali uses plenty of RAM, and Ubuntu is often the first to add new graphical features, which some YouTubers claim pushed them to Linux Mint to avoid slowdown. However, there are many Linux distributions that use effective, but not cutting edge, features sparingly. The result varies, but can include extending the speed of a newer computer, or restoring an older computer to useability. I once used a "micro-distro" to speed up a clean room computer at a pharmaceutical firm, which was using a new version of windows on a fairly old hardware set up. It worked, but that  he IT department disliked it. Anyway, result is that certain distributions can extend your laptop's lifespan. The Laptop I'm writing this on is 8 years old, and runs Bodhi Linux, and works just fine.

Second, Linux is very adaptable. Most Linux distributions have some form of customisation available at installation. Some are also designed with specific features and applications in mind. As mentioned, Kali Linux is made with Penetration Testing in mind. That is, it comes with a series of hacking tools, and a lot of personal security features. Ubuntu has Edubuntu, a version compiled with teachers in mind, coming with a bundle of teaching software. It also has several other specialist versions, such as Ubuntu Studio for creatives, and Ubuntu Kylin, which is optimised for Chinese speakers. The Extreme examples of this are Arch Linux and Gentoo; The first of these examples requires you to make every minor decision about its form when you install it. The Second is specifically sold on the same grounds, with the specific aim of optimising your computer for a specific application. 

Finally, as highlighted earlier; by and large Linux is Free. MacOS is also a UNIX offshoot, and so is BSD, which you've never heard of, and I refuse to comment further on. Linux gives its Kernel (computer word for ... well something) to a series of free operating systems, most of which are compatible with a huge range of hardware. Ubuntu makes its money byhadoing contract work for comp hanies that want to run one of its flavours on their systems. Raspberry Pi sells their hardware, but their OS is freely available. In short, this means you have options. Therefore, you can easily experiment and, as long as you make backups, you won't need to worry about losing anything.\end{document}
